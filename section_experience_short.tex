% Awesome CV LaTeX Template
%
% This template has been downloaded from:
% https://github.com/huajh/huajh-awesome-latex-cv
%
% Author:
% Junhao Hua


%Section: Work Experience at the top
\sectionTitle{实习/项目经历}{\faCode}
 
\begin{experiences}
			
 \experience
    {2017年9月} {水下焊接机器人建模与控制研究}{ 哈尔滨工业大学}{博士研究课题}
{2014年9月} {
          \begin{itemize}
          	\item 设计完成了\emph{国内首款核电站水下焊接机器人},采用矢量布置的八推进器实现机器人水下全姿态运动,搭载三自由度移动平台完成水下焊接作业;
            \item 完成了水下焊接机器人软硬件设计,采用远程遥操作与机器人自主运动控制相结合的控制结构,水下控制系统采用搭载Linux内核的ARM芯片作为主控芯片,水面控制系统采用机器人操作系统(ROS);
            \item 提出了\emph{多区域划分定位算法(MRDL)},结合高度计与姿态航向参考系统实现受限水域水下定位。
            \item 建立了\emph{水下推进器推力预测模型},利用\emph{高斯过程回归}对推进器类空化效应进行预测,建立精确的推进器推力模型;
            \item 提出了\emph{水下机器人变质心补偿算法},对三自由度移动平台作业过程中的变质心特性进行补偿,实现高精度的焊接稳定控制;
            \item \faGithub: 
            \link{https://github.com/thinkexist1989/Underwater-Welding-Vehicle}{Underwater-Welding-Vehicle}, 
            \link{https://github.com/thinkexist1989/ROVControl}{ROVControl},
            \link{https://github.com/thinkexist1989/ThrustExpr}{Thrust Experiment},
            \link{https://github.com/thinkexist1989/Program_on_Pompano}{Program on Pompano}. 
                
          \end{itemize}
        }
        {水下焊接机器人, ROS,多区域划分定位算法, 高斯过程回归, 变质心补偿算法, Linux, C/C++, Qt, MFC, CMAKE}
  \emptySeparator
  \experience
    {2018年1月} {高精度石墨烯生物检测系统}{哈尔滨工业大学}{ 独立开发}
    {2017年9月}    {
		\begin{itemize}
			\item 采用凌特(Linear Tech) 高精度运放芯片LT1462实现10uA电流转换电压与电压信号放大;
			\item 采用德州仪器(TI)16位高精度DA芯片DAC8830与LT1462实现10mV高精度稳压输出; 
			\item 采用德州仪器(TI)4路24位高精度AD芯片ADS1274与THS4524差动运放实现高精度电压采集;
			\item 采用意法半导体(ST) STM32F407芯片实现数据处理、通讯处理和实时显示,利用HAL库实现程序快速配置与封装;                    
			\item 完成4层PCB印制电路板设计,并利用Qt编写PC端数据采集与实时曲线显示。                                                                                  
		\end{itemize}
		}
		{LT1462, DAC8830, ADS1274, THS4524, STM32F407, HAL库, Altium Designer, Qt, ChartDirector}

	\emptySeparator
	\experience
	{2014年7月} {室内助行机器人导航研究}{哈尔滨工业大学}{硕士毕业设计}
	{2012年9月}    {
	   	\begin{itemize}
	   		\item 采用稀疏超声波阵列方式,实现了\emph{超声波网络定位系统设计},在满足精度要求前提下极大降低了超声波阵列的布置密度;
	   		\item 利用编码器与电子罗盘的航迹推算组合实现了大范围长距离定位;
	   		\item 利用\emph{双层卡尔曼滤波}实现航迹推算和超声波网络定位的数据融合,避免长距离误差积累。
	   		\item 完成导航系统软硬件设计,采用ARM作为主控芯片,搭载Linux内核,利用Qt实现GUI设计与多线程任务处理。 \faGithub: \link{https://github.com/thinkexist1989/NaviOS}{NaviOS}.	 		
	   		
	   	\end{itemize}
	}
	{助行机器人, 超声波网络定位, 航迹推算, ARM, Linux, Qt}
	
	\emptySeparator
	\experience
	{2013年5月} {喷管延伸段延伸机构功能及载荷试验}{哈尔滨工业大学}{设计全程参与、试验独立完成}
	{2012年7月}    {
		\begin{itemize}
			\item 航天一院委托哈工大设计研制的\emph{喷管延伸机构试验装置},本人独立完成功能及载荷试验;
			\item 电控系统以研华PCI-1784数据采集卡为核心,基于Labview实现底层控制与人机交互,本人完成对电控部分的程序编写、优化及修改;
			\item 独立完成喷管延伸段延伸机构的失重试验,载荷试验及超载试验。
		\end{itemize}
	}
	{二级火箭推进器, 喷管延伸机构, PCI-1784, Labview, GUI}

	\emptySeparator
	\experience
	{2012年7月} {直驱集成一体式电液推力装置设计}{哈尔滨工业大学}{本科毕业设计}
	{2011年12月}    {
	 	\begin{itemize}
	 		\item 对\emph{直驱集成一体式液压推力装置}进行了系统设计,其中包括液压系统设计、液压元件选择、阀块设计、液压缸设计、电液推杆整体结构设计;
	 		\item 利用AutoCAD软件绘制了液压系统原理图、阀块零件图、阀块装配图、液压缸装配图、液压缸前端盖零件图、电液推杆总装图、连接板零件图、联轴箱零件图和外观图;
	 		\item 对直驱集成一体式液压推力装置进行了电控部分设计,采用西门子PLC作为伺服控制器。
	 	\end{itemize}
	}
	{AutoCAD, 直接驱动, 交流伺服电机, 闭式回路, 液压缸}
		
\end{experiences}
